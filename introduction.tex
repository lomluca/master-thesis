\chapter{Introduzione generale}

\section{Principi generali}
Il problema della determinazione della pressione barometrica dell'atmosfera di
Giove non ha ricevuto finora una soluzione soddisfacente, per l'elementare
motivo che il pianeta suddetto si trova ad una distanza tale che i mezzi attuali
non consentono di eseguire una misura diretta.

Conoscendo per{\`o} con grande precisione le orbite dei satelliti principali di
Giove, e segnatamente le orbite dei satelliti medicei, {\`e} possibile eseguire
delle misure indirette, che fanno ricorso alla nota formula \cite{gal}:
\[
\Phi = K\frac{\Xi^2 +\Psi\ped{max}}{1+\gei\Omega}
\]
dove le varie grandezze hanno i seguenti significati:
\begin{enumerate}
\item
$\Phi$ angolo di rivoluzione del satellite in radianti se $K=1$, in gradi se
$K=180/\pi$;
\item
$\Xi$ eccentricit{\`a} dell'orbita del satellite; questa {\`e} una grandezza priva
di dimensioni;
\item
$\Psi\ped{max}$ rapporto fra il semiasse maggiore ed il semiasse minore
dell'orbita del satellite, nelle condizioni di massima eccentricit{\`a};
poich{\'e} le dimensioni di ciascun semiasse sono $[l]=\unit{km}$, la grandezza
$\Psi\ped{max}$ {\`e} adimensionata;
\item
$\Omega$ velocit{\`a} istantanea di rotazione; si ricorda che {\`e} $[\Omega]=%
\unit{rad}\unit{s}^{-1}$;
\item bisogna ancora ricordarsi che $10^{-6}\unit{m}$ equivalgono a
1\unit{\micro m}.
\end{enumerate}
%

Le grandezze in gioco sono evidenziate nella figura \ref{fig1}.
\begin{figure}[ht]\centering
\setlength{\unitlength}{0.01\textwidth}
\begin{picture}(40,30)(30,0)
\put(50,15){\circle{20}}
\put(47,15){\circle*{1}}
\put(30,0){\line(0,1){30}}
\put(30,30){\line(1,0){40}}
\put(70,30){\line(0,-1){30}}
\put(70,0){\line(-1,0){40}}
\end{picture}
\caption{Orbita del generico satellite; si noti l'eccentricit� dell'orbita rispetto al pianeta.\label{fig1}}
\end{figure}

Per misurare le grandezze che compaiono in questa formula {\`e} necessario
ricorrere ad un pirometro con una resistenza di 120\unit{M\ohm}, altrimenti gli
errori di misura sono troppo grandi, ed i risultati completamente falsati.

\section{I satelliti medicei}
I satelliti medicei, come noto, sono quattro ed hanno dei periodi di rivoluzione
attorno al pianeta Giove che vanno dai sette giorni alle tre settimane.

Essi furono per la prima volta osservati da uno dei candidati mentre
sperimentava l'efficacia del tubo occhiale che aveva appena inventato
rielaborando una idea sentita di seconda mano da un viaggiatore appena arrivato
dai Paesi Bassi.